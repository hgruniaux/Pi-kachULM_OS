\documentclass[french, 12pt]{article}
\usepackage[a4paper, top=1cm, bottom=2cm, left=1cm, right=1cm]{geometry}

\usepackage[mono=false]{libertine}
\usepackage[french]{babel}
\usepackage{hyperref}
\usepackage{textcomp}

\title{Pi-kachULM\_OS}
% \author{Émile Sauvat, Hubert Gruniaux \& Gabriel Desfrene}  % Mettre en bas de page.
\date{}

\begin{document}
\maketitle

% \tableofcontents

\section{Présentation}
\subsection{Objectif}
Nous nous sommes fixés comme objectif premier la réalisation d'un système
d'exploitation en C++ s'exécutant sur une
\href{https://www.raspberrypi.com/products/raspberry-pi-3-model-b-plus/}{\emph{Raspberry Pi 3B+}}.
Ce choix était principalement motivé par le support de cette plateforme par
l'émulateur \href{https://www.qemu.org/}{\emph{QEMU}}.

Finalement, par curiosité et pour le fun\texttrademark, nous av ons finalement
choisi de supporter plusieurs modèle de \emph{Raspberry Pi} :
\begin{itemize}
    \item \href{https://www.raspberrypi.com/products/raspberry-pi-3-model-b-plus/}{\emph{Raspberry Pi 3B+}}
    \item \href{https://www.raspberrypi.com/products/raspberry-pi-4-model-b/}{\emph{Raspberry Pi 4 Model B}}
    \item \href{https://www.raspberrypi.com/products/compute-module-4/?variant=raspberry-pi-cm4001000}{\emph{Raspberry Pi Compute Module 4}}
\end{itemize}

La totalité de ce qu'il suit à donc été testé et conçu afin d'être exécuté sur
l'un de ces modèles. Ce projet compte environ XXX lignes de codes de C++, xx
lignes de C et xx lignes d'assembleur. La totalité du code est disponible à
l'adresse suivante : \url{https://github.com/hgruniaux/Pi-kachULM_OS}

\subsection{Reconnaissance dynamique du matériel}
Afin de supporter plusieurs machines, qui ne partagent pas toutes les mêmes
caractéristiques et spécificités nous avons besoin d'un moyen de les déterminer
dynamiquement, lors du démarrage du noyau. Pour ce faire, le noyau utilise et le
\textit{DeviceTree} donné par le \textit{bootloader}. Nous avons donc développé une
bibliothèque, \texttt{libdevice-tree} permettant d'interpreter cette structure.

Un \textit{DeviceTree} est, comme son nom l'indique, un arbre, compilé sous la
forme d'un fichier binaire, qui est chargé par le \textit{bootloader} lors du
démarrage. Cet arbre est rempli par ce dernier, et informe le noyau du matériel
présent. Une documentation est disponible sur cette
\href{https://www.devicetree.org/specifications/}{page}.

\section{Modèle mémoire}
\subsection{Vue de la mémoire}
Lors du démarrage du système d'exploitation, ce dernier récupère la taille de la
mémoire disponible et s'occupe d'initialiser le \textit{Memory Management Unit (MMU)}
pour que le noyau puisse s'abstraire de la mémoire physique. Cette configuration
est effectuée au sein du fichier \texttt{mmu\_init.cpp}. Les pages ont une
taille fixée à 4~kio.

On utilise ici à bon escient les spécificités du \textit{MMU} de la plateforme
ARM. Cette dernière nous permet d'avoir deux espaces d'adressage complètement
distincts :

\begin{itemize}
    \item La mémoire des processus est dans l'espace d'adresse \texttt{0x0000000000000000} à \texttt{0x0000ffffffffffff}.
    \item La mémoire du noyau est, elle, dans l'espace d'adresse \texttt{0xffff000000000000} à \texttt{0xffffffffffffffff}.
\end{itemize}

% Do a scheme of the memory mapping

\subsection{Allocateurs de pages}
Une fois que le noyau est passé en mémoire virtuelle, il nous faut un moyen
de garder une trace des pages physiques libres ou utilisées par le noyau. Pour
ce faire, nous avons implémenté deux allocateur de pages :
\begin{itemize}
    \item \texttt{page\_alloc.hpp} s'occupe de l'allocation de la grande
          majorité des pages physiques de la mémoire. Cette allocation est
          réalisée de manière optimisée par un arbre binaire. % Add some notes here ? Develop the algorithm ?
    \item \texttt{contiguous\_page\_alloc.hpp} travaille sur une petite portion
          de la mémoire physique et s'occupe d'allouée une liste de pages
          contiguës. Cette allocation est utilisée pour les \textit{buffers}
          utilisés lors des transferts DMA.
\end{itemize}

\subsection{Allocateurs de mémoire}
À l'aide de l'allocateur de page principal et en modifiant les tables de
translation du \textit{MMU} du noyaux


\subsection{Système de fichier}
Le RamFs et les fonctionnalités (FAT)

\section{\textit{Drivers}}
\subsection{GPIO}
\subsection{Uart}
\subsection{Mini Uart}
\subsection{Mailbox}
\subsection{Contrôleur IRQ}
\subsection{Contrôleur DMA}
\subsection{System Timer}
\subsection{Timer ARM}

\section{Processus}
\subsection{Hiérarchie}
Support des processus utilisateurs et noyaux leur \texttt{libc}.
Pas de diff entre process et thread.

\subsection{Appels systèmes}
Liste des appels systèmes possibles (sycalls de \texttt{libsyscall})

\section{Ordonnanceur}
\subsection{Algorithme et Priorités}
Comment l'ordonnateur fonctionne, qu'est-ce qu'il est capable de faire ?

\subsection{Mise en pause de Processus}
Comment et quand les processus font dodo ?

\section{Écran et Fenêtres}
\subsection{\textit{Framebuffer}}
Rapidement comment on affiche des choses à l'écran.

\subsection{Gestionnaire de fenêtres}
Fonctionnalités et choix d'implementation (Processus Noyaux).

\section{Clavier PS/2}
Clavier PS2.

\section{Conclusion}
Quoi qu'on dit là ?


\end{document}
